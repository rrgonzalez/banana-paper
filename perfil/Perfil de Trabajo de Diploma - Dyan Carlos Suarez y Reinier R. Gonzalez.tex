\documentclass[11pt]{article}
\usepackage[margin=1in]{geometry}
%\usepackage{}

\linespread{1.3}

\title{Perfil de Trabajo de Diploma}
\date{Curso: 2015-2016}

\begin{document}

\maketitle

\noindent\textbf{T\'itulo:}
\newline
``Componente para la fusi\'on de im\'agenes PET con CT''.

\hspace{0pt} \\
\textbf{Autores:}
\begin{itemize}
\item Dyan Carlos Yanes Su\'arez
\item Reinier Rodr\'iguez Gonz\'alez
\end{itemize}

\hspace{0pt} \\
\textbf{Tutores:}
\begin{itemize}
\item Ing. Yarianna Castellanos Del Toro
\item Ing. Maikel S\'anchez Dieguez
\end{itemize}

\hspace{0pt} \\
\textbf{Clasificaci\'on: }
\newline
Investigaci\'on + Desarrollo (Ciclo completo)

\hspace{0pt} \\
\textbf{Clasificaci\'on del \'area de desarrollo: }
\newline
Procesamiento de Im\'agenes M\'edicas.

\hspace{0pt} \\
\textbf{S\'intesis de los Tutores: }
\\
\textbf{Yarianna Castellanos Del Toro:} Graduada de Ingeniera en Ciencias Inform\'aticas, egresada de la UCI en el a\~no 2014. Se desempe\~na como especialista del Departamento de Desarrollo de Aplicaciones del Centro de Inform\'atica M\'edica (CESIM) de la UCI. Es Administradora de Calidad de la soluci\'on PACS-RIS. Correo electr\'onico: ydeltoro@uci.cu
\\
\textbf{Ing. Maikel S\'anchez Dieguez:} Graduado de Ingeniero en Ciencias Inform\'aticas, egresado de la Universidad de las Ciencias Inform\'aticas (UCI) en el a\~no 2010. Especialista del Departamento de Desarrollo de Componentes del CESIM, de la UCI. Se desempe\~na como l\'ider del equipo de desarrollo del sistema PACS. Correo Electr\'onico: msdieguez@uci.cu

\hspace{0pt} \\
\textbf{Problema a resolver:}
\newline
El diagn\'ostico por imagen, es una de las t\'ecnicas m\'as utilizadas en la actualidad para detectar enfermedades y emitir tratamientos. Con el paso de los a\~nos t\'ecnicas como la Resonancia Magn\'etica (MR), la Tomograf\'ia Computarizada (CT), los Ultrasonidos (US),  las Angiograf\'ias (XA), entre otras han ido evolucionando para dar lugar a nuevas t\'ecnicas como la Tomograf\'ia por Emisi\'on de Positrones (PET) y Resonancia Magn\'etica Funcional (MRI).

Cada una de estas t\'ecnicas o modalidades tienes prop\'ositos espec\'ificos y brindan informaci\'on a los m\'edicos para que puedan emitir un diagn\'ostico. Para que un radi\'ologo pueda emitir un diagn\'ostico acertado necesita la mayor cantidad de informaci\'on posible, por lo que se hace necesario poder interpretar im\'agenes estructurales como las CT e im\'agenes funcionales como las PET. Actualmente esta interpretaci\'on se hace de distintas formas: mirando las im\'agenes obtenidas de los equipos directamente, con software especializados en fusionar estas im\'agenes, que tienen un alto costo, o mediante equipos m\'edicos especializados en estos prop\'ositos.

La posibilidad de combinar estos estudios permite obtener im\'agenes muy claras que combinan las im\'agenes anat\'omicas que muestra el CT con las de funci\'on celular (o metabolismo) que proporciona el PET. Esta fusi\'on de las dos t\'ecnicas hace que el estudio sea mucho m\'as confiable, y f\'acil de interpretar.  La combinaci\'on de estos estudios permite:
\begin{itemize}
\item Detectar c\'ancer.
\item Determinar si un c\'ancer se ha diseminado en el cuerpo.
\item Evaluar la eficacia de un plan de tratamiento, tal como la terapia de c\'ancer.
\item Determinar el flujo sangu\'ineo hacia el m\'usculo card\'iaco.
\item Determinar los efectos de un ataque card\'iaco, o infarto del miocardio, en \'areas del coraz\'on.
\item Identificar \'areas del m\'usculo card\'iaco que se beneficiar\'ian mediante un procedimiento tal como angioplastia o cirug\'ia de bypass coronario (en combinaci\'on con un estudio de perfusi\'on mioc\'ardica).
\item Evaluar anomal\'ias cerebrales, tales como tumores, des\'ordenes de la memoria convulsiones y otros des\'ordenes del sistema central nervioso.
\item Esquematizar el cerebro humano normal y la funci\'on card\'iaca.
\end{itemize}

\vspace{8pt}

\noindent\textbf{Objetivo:}
\newline
Desarrollar un componente de software que permita la fusi\'on de im\'agenes de CT con PET.

\vspace{25pt}

\noindent\textbf{Tareas a cumplir por los estudiantes:}
\begin{enumerate}
\item Estudiar los fundamentos de las Im\'agenes M\'edicas Digitales.
\item Examinar los estudios provenientes de equipos de Tomograf\'ia Computarizada y Tomograf\'ia por Emisi\'on de Positrones.
\item Explorar las t\'ecnicas de registro y co-registro de im\'agenes m\'edicas.
\item Investigar las principales t\'ecnicas empleadas en la fusi\'on de im\'agenes m\'edicas y seleccionar la m\'as conveniente, para su implementaci\'on.
\item Seleccionar la tecnolog\'ia adecuada para el trabajo: framework, herramienta de control de versiones, etc\ldots
\item Implementar un algoritmo para la fusi\'on de im\'agenes de CT y PET.
\item Validar el resultado obtenido.
\item Documentar la fase del proceso de desarrollo del componente.
\item Integrar el componente desarrollado a la soluci\'on del visor de im\'agenes m\'edicas del CESIM.
\end{enumerate}

\vspace{8pt}

\noindent\textbf{Resultado esperado:}
Como resultado final, se espera una biblioteca de enlaces din\'amicos (dll), donde se agrupen los m\'etodos necesarios para la fusi\'on de im\'agenes de CT y PET.

%\vspace{8pt}

%\begin{table}[ht]
%\textbf{Cronograma:}\\
%\begin{tabular}{|c|c|c|c|c|}
%\hline
%\textbf{No.} & \textbf{Descripci\'on de la Tarea/Subtarea} & \textbf{Responsable} & \textbf{Fecha Inicio} & \textbf{Fecha Fin} \\
%\hline
%\end{tabular}
%\end{table}

\vspace{1.7in}

\begin{tabular}[htbc]{c @{\hskip 0.8in} c}
\makebox[2.5in]{\hrulefill} & \makebox[2.5in]{\hrulefill} \\
Dyan Carlos Yanes Su\'arez & Reinier Rodr\'iguez Gonz\'alez \\
Autor & Autor
\end{tabular}

\vspace{1.0in}

\begin{tabular}[htbc]{c @{\hskip 0.8in} c}
\makebox[2.5in]{\hrulefill} & \makebox[2.5in]{\hrulefill} \\
Ing. Yarianna Castellanos Del Toro & Ing. Maikel S\'anchez Dieguez \\
Tutora & Tutor
\end{tabular}

\end{document}
